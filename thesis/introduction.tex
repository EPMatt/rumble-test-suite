
\chapter{Introduction}
\label{chapter:introduction}
The increasing amount of data available to process, as well as the ever-growing discrepancy between storage capacity, throughput and latency, has forced the database community to come up with new querying paradigms in the last two decades. Data became nested and heterogeneous (JSON), and is increasingly processed in parallel (Spark). In order to make querying more efficient and accessible, Rumble \cite{RumblePaper} is an engine that automatically runs queries on semi-structured and unstructured documents on top of Spark, using the JSONiq language. 

JSONiq \cite{JSONIQORG}is a functional and declarative language that addresses these problems with its most useful FLWOR expression, which is the more flexible counterpart of SQL’s SELECT FROM WHERE. It inherits 95\% of its features from XQuery, a W3C standard.

The XQuery/XPath 3.* Test Suite (QT3TS) \cite{TestSuite} provides a set of tests with over 30000 test cases designed to demonstrate the interoperability of W3C XML Query Language, version 3.0 and W3C XML Path Language implementations.

The high-level idea of this work is to implement a Test Driver that can directly use QT3TS in order to test and verify Rumble implementation. The implementation of the Test Driver will be carried out gradually through iterations. We will analyze the output of each iteration to measure the implementation success. Test Driver and its architecture will gradually evolve and improve through several phases of implementation. 